\chapter*{Publications}\label{chap:publis}

\begin{itemize}
\item T. Moulard, F. Lamiraux et O. Stasse. Trajectory Following for
  Legged Robots. In \emph{International Conference on Biomedical
    Robotics and Biomechatronics (BioRob'2012)}, Rome, Italie, juin
  2012.
\item L. Baudouin, T. Moulard, N. Perrin, F. Lamiraux, O. Stasse et
  E. Yoshida. Real-time Replanning Using 3D Environment for Humanoid
  Robot. In \emph{IEEE-RAS International Conference on Humanoid Robots
    (HUMANOIDS 2011)}, Bled, Slovénie, octobre 2011.
\item T. Moulard. Using numerical optimization in path planning,
  application to humanoid robot walk planning. In \emph{Journées
    Nationales de la Robotique Humanoïde}, Toulouse, France, avril
  2011.
\item T. Moulard. Coordinate Frames for Humanoids Robots. Ros
  Enhancement Proposal 120, novembre 2011. URL
  \url{http://www.ros.org/reps/rep-0120.html}
\end{itemize}


\chapter*{Contributions logicielles principales}\label{chap:soft}

\begin{itemize}
\item T. Moulard, F. Lamiraux, P.-B. Wieber et O. Stasse. RobOptim: un
  framework pour l'optimisation numérique en robotique. LGPL-3.0. URL
  \url{http://www.roboptim.net/}
\item T. Moulard. ViSP Tracker: un composant robotique ROS pour le
  suivi d'objet en temps réel se fondant sur les algorithmes de la
  bibliothèque ViSP conçu par l'équipe LAGADIC. BSD. URL
  \url{http://ros.org/wiki/visp_tracker}
\item T. Moulard. Motion Analysis Mocap: un composant robotique ROS
  pour le suivi temps réel d'objet par la capture de mouvements fondé
  sur le système Cortex de la société Motion Analysis. BSD. URL
  \url{http://ros.org/wiki/motion_analysis_mocap}
\item T. Moulard. RCPDF: une description unifiée des zones de contact
  autorisées sur un robot. BSD. URL
  \url{http://ros.org/wiki/robot_contact_point}
\item T. Moulard, David Lu. RobotModelPy: un module Python permettant
  le chargement de modèles au format URDF. BSD. URL
  \url{http://ros.org/wiki/robot_model_py}
\item F. Keith, T. Moulard, Romeo: modèle du robot Romeo d'Aldebaran
  Robotics au format URDF. BSD. URL \url{http://ros.org/wiki/romeo}
\end{itemize}
