\chapter{Primitives de mouvement}
\label{chap:primitive}

\epigraph{Text text text text text text text text text text text text
  text text text text text text text text text text text text text
  text text text text text text text text text text text text text
  text text text text text text text text text text text text text
  text text text text}{Some Author}
\clearpage

\section{Problématique}

Le chapitre précédent a introduit la possibilité d'asservir une
trajectoire de marche sur un robot humanoïde. Cependant, les tâches
accomplies par les robots humanoïdes ne se limitent pas à la
locomotion et il est intéressant de se demander s'il est possible de
combiner aux tâches de navigation d'autres tâches asservies par les
données capteurs afin de réaliser des comportements complexes. De
manière indirecte, la question qui se pose ici est celle de la limite
à placer entre d'une part raisonnement numérique et d'autre part
raisonnement logique. Ce problème récurrent de la robotique et pour
lequel il n'existe pas, de consensus au sein de la communauté trouve
ici une solution élégante. En effet, un contrôleur fondé sur le
paradigme de la pile de tâche ne se limite pas à une description
d'objectifs ou de contraintes robotiques dans des espaces plus
naturels que l'espace des configurations ou l'espace cartésien, il
ouvre surtout la voie à des méchanismes de supervision décidant à quel
moment insérer telle ou telle tâche à un niveau de priorité donné. La
pile de tâche réalise donc la jonction entre d'une part, le monde du
calcul numérique puisque la finalité du système est de calculer la
commande du système et d'autre part le monde de la logique dans lequel
les utilisateurs humains souhaitent exprimer leur \emph{desiderata} au
système robotique: va jusqu'à la cuisine, apportes moi la bouteille,
ouvres la porte, etc. Il est clair que ce type d'ordre nécessite la
résolution d'abstractions et que la logique mathématique semble être
un moyen particulièrement adapté pour y parvenir. Ces méchanismes de
décision de haut niveau sont appelés ``superviseurs'' et ont pour
objectif d'instantier et d'orchestrer tous les composants d'une
architecture robotique. Dans le cadre de notre architecture, il faut
donc pouvoir instantier des piles de tâches à partir d'une description
de haut niveau des ordres passés au robot. Ce chapitre propose un
langage de description des tâches permettant de faire le lien entre
représentation logique et représentation numérique.


Nous allons commencer par détailler l'état de l'Art et en particulier
les autres travaux aillant trait aux architectures robotiques haut
niveau et à l'ordonnancement de tâches ainsi que plus généralement aux
applications robotiques complexes. Dans un second temps, le langage de
description sera décrit et quelques scénarii types seront
démontrés. Enfin, nous nous concentrerons sur les problèmes
d'asservissement posés par les robots humanoïdes avant de conclure.


\section{\'Etat de l'Art}

\section{Description d'un mouvement robotique complexe}
\subsection{Primitive de mouvement}
\subsection{Langage de description}
\subsection{Primitive de locomotion}
\subsection{Primitive de manipulation}
\subsection{Asservissement sur un capteur et fermeture de la boucle}

\section{Scénarii de mouvements}
\subsection{Locomotion simple}

\begin{figure}
  \begin{center}
\begin{verbatim}
duration: 200 # durée complète du mouvement (secondes)

# Éléments de mouvement
motion:
  # Primitive de locomotion
  - walk:
      interval: [0, 200] # Date de début et de fin de la primitive

      # Pile de pas
      footsteps:
      - {x: 0.15, y: -0.19, theta: 0., slide1: 0., slide2: -0.76}
      - {x: 0.15, y: +0.19, theta: 0.1, slide1: -1.00, slide2: -0.76}
      # etc.
\end{verbatim}
  \end{center}
  \caption{Plan de mouvement pour une séquence de marche non-asservie.}
\end{figure}

\subsection{Locomotion asservie}

\begin{figure}
  \begin{center}
\begin{verbatim}
duration: 200 # durée complète du mouvement (secondes)

# correction maximum autorisée sur un pas (x, y en mètre, theta en radians)
maximum-correction-per-step: {x: 0.02, y: 0.02, theta: 0.05}

# Éléments de mouvement
motion:
  # Primitive de locomotion
  - walk:
      interval: [0, 200] # Date de début et de fin de la primitive

      # Pile de pas
      footsteps:
      - {x: 0.15, y: -0.19, theta: 0., slide1: 0., slide2: -0.76}
      - {x: 0.15, y: +0.19, theta: 0.1, slide1: -1.00, slide2: -0.76}
      # etc.

# Asservissement des tâches
control:
  # Asservissement via le système de capture de mouvements.
  - mocap:
      weight: 1.
      tracked-body: left-ankle
      perceived-body: left-foot
\end{verbatim}
  \end{center}
  \caption{Plan de mouvement pour une séquence de marche asservie sur
    un système de localisation -- ici un système de capture de
    mouvement --.}
\end{figure}


\subsection{Scénario complexe avec tâche d'atteinte et asservissement corps complet}

\section{De la difficulté à localiser un robot humanoïde}
\subsection{Forces et limites du robot humanoïde HRP-2}
\subsection{Vision pour la robotique humanoïde}
\paragraph{Difficultés classiques de la vision}
\paragraph{Stabilisation de la tête}
\paragraph{Pourquoi un robot n'est pas une caméra volante}
\paragraph{Résultats de la localisation utilisant la vision sur le robot humanoïde HRP-2}

\section{Résultats}

\section{Conclusion}
