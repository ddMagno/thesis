\chapter[Optimisation]{Optimisation et généricité}
\label{chap:roboptim}

\epigraph{Text text text text text text text text text text text text
  text text text text text text text text text text text text text
  text text text text text text text text text text text text text
  text text text text text text text text text text text text text
  text text text text}{Some Author}
\clearpage

\section[Optimisation]{Optimisation numérique et robotique}

\subsection{Introduction à l'optimisation numérique}

\lettrine[lines=2, lraise=0.1, nindent=0em, slope=-.5em]%
{L}{a} robotique s'appuie sur différents outils mathématiques
pour résoudre les problèmes auxquels elle se trouve confrontée. Parmi
ces outils, l'optimisation numérique figure à une place de choix.


L'optimisation numérique est une branche des mathématiques permettant
de choisir automatiquement la meilleure solution parmi un ensemble de
solutions possibles. Les applications possibles couvrent des domaines
variés de la recherche opérationnelle aux statistiques et bien sûr la
robotique. Dans ce domaine particulier, l'optimisation numérique
permets par exemple d'optimiser un mouvement pour en améliorer
certains caractéristiques ou bien encore trouver la prochaine commande
à envoyer aux moteurs du robot afin de réaliser une tâche donnée.



Résoudre un problème d'optimisation numérique signifie trouver les
meilleurs paramètres solutionnant le problème. Par meilleurs, il faut
entendre les paramètres minimisant la valeur d'une fonction de
coût. Intuitivement, on peut se représenter cette fonction comme un
indicateur de la ``qualité'' de la solution trouvée. Une solution
présentant un coût fort sera donc peu satisfaisante tandis qu'une
solution présentant un coût faible sera, elle, très attrayante. Tout
le raisonement est ensuite fondé sur la linéarité -- au moins locale!
-- de cette fonction de coût. Plus simplement, près d'une mauvaise
solution, il n'y aura que des solutions légèrement meilleures et
légèrement pires, mais jamais trop différentes. De ce fait, il
``suffit'' de chercher dans quelle direction aller pour se diriger
vers les meilleurs solutions pour finir par les trouver. Cette
approche implique une limitation majeure: si la fonction de coût n'est
pas convexe, on ne trouvera pas forcément la meilleure solution
globale -- parmi toutes les solutions -- mais ma meilleure solution
locale. C'est à dire qu'au voisinage de cette solution, toutes les
autres solutions sont de moins bonnes qualités. Dans cette situation,
tous les solveurs arrêtent leur recherche, mais rien ne garantit
qu'ailleurs il n'existe pas une autre solution de meilleure qualité.


Ce raisonnement permets de trouver les meilleurs paramètres pour un
problème donné, mais il est rare qu'une fonction de coût seule puisse
capturer complètement un problème donné. En fonctionnant de manière
indépendante du problème sous-jacent, on confère aux solveurs une
grande polyvalence. Mais par là même, il devient nécessaire
d'expliciter sous quelles conditions une solution au problème donné
est acceptable. Prenons un exemple: pour aller d'un point $A$ à un
point $B$ le plus rapidement possible, se déplacer avec une vitesse
infinie est sans aucun doute la solution optimale. Cependant, une
telle réponse ne présente guère d'intérêt dans la mesure où aucun
système physique ne sera en mesure de l'executer.


Deux stratégies peuvent alors être choisies. La première est
d'exprimer directement dans la fonction de coût cette information
supplémentaire. Dans le cadre de l'exemple précédemment cité, on
pourrait concevoir une fonction de coût réalisant la somme du temps de
déplacement et de la vitesse moyenne du robot. De ce fait, une vitesse
trop grande pénalise le résultat et par ce biais, le solveur tentera
plutôt de transformer la trajectoire géométrique plutôt que de jouer
sur la vitesse de déplacement du système. La limite de cette approche
est simple à comprendre: au fur et à mesure que les biais s'accumulent
-- le plus souvent en sommant les uns aux autres --, le coût initial
peut se retrouver ignoré par le solveur car numériquement
négligeable. La fonction de coût n'a alors plus aucune justification
physique et ses variations peuvent devenir de plus en plus difficiles
à analyser pour le solveur.  Une seconde solution consiste à ajouter
des contraintes au problème. Ces contraintes sont exprimées sous la
forme d'équation ou d'inéquation déterminant si une solution donnée
est une solution acceptable. En robotique, des contraintes très
courantes sont les bornes sur les butées articulaires. Il est rare sur
un robot que les axes puissent bouger de manière totalement libre. De
ce fait, il est courant d'ajouter des contraintes inégalités
spécifiant que tel ou tel valeur de joint doit rester dans un
intervalle donné.


En exprimant comment résoudre un problème pratique via la construction
d'une fonction de coût $f$ et de contraintes $g_i > 0$,
\textbf{modélise} le problème afin qu'il soit soluble par optimisation
numérique. Ce procédé n'est pas trivial car la qualité de la
modélisation impacte lourdement à la fois les temps de calcul et la
qualité du résultat final.



\subsection{Modélisation mathématique d'un problème}


Résoudre un problème d'optimisation revient à trouver une solution pour:

\begin{equation}
  \min_{\mathbf{x} \in \mathbb{R}^n} f(x) \text{ sous la contrainte } \mathbf{x} \in \mathbf{X}
\end{equation}

où $f : \mathbb{R}^n \mapsto \mathbb{R}$ est la fonction de coût et
$\mathbf{X} \subset \mathbb{R}^n$ est l'espace des solutions
admissibles. Cet espace est habituellement défini par un ensemble de
contraintes égalités et inégalités:

\begin{equation}
  \mathbf{X} \equiv \left\{
  \begin{array}{l l}
    c_i (x) = 0    & \quad i \in \xi \\
    c_j (x) \leq 0 & \quad j \in \nu \\
  \end{array} \right.
\end{equation}

$c_i$, $c_j$ sont respectivement les ensembles de contraintes égalités
et inégalités. $i$ et $j$ des indices identifiant les contraintes.


L'optimisation numérique a pour objectif de développer des stratégies
-- algorithmes -- pouvant déterminer $\mathbf{x} \in \mathbf{X}$
minimisant les valeurs de la fonction $f$.


En absence d'argument fort tel que la convexité de $f$ ainsi que des
contraintes $c_i, i \in \xi \cup \nu$, la solution $\mathbf{x}$
fournie par le solveur peut être un minimum local \index{minimum
  local}.


En effet, la plupart des méthodes d'optimisation tentent de trouver un
minimum en raffinant itérativement une solution à partir d'un candidat
initial $\mathbf{x_{\text{init}}} \in \mathbf{X}$. Le solveur, dès
lors, nécessite un critère d'arrêt à ce processus itératif. Ce critère
est donné par les conditions de Karush-Kuhn-Tucker \index{conditions
  de Karush-Kuhn-Tucker} ou conditions KKT.

\begin{mydef}\label{def:chap1_kkt}
Si $\mathbf{x} \in \mathbf{X}$ est un minimum local, alors il existe
$\lambda_i, i \in \xi$ et $\mu_j, j \in \nu$ des constantes non-nulles
appelées multiplicateurs KKT tels que:
%
\begin{description}
\item[\textbf{Critère de stabilité}] \begin{equation}
  \nabla f(x) + \sum_{i \in \xi} \lambda_i c_i(x) + \sum_{j \in \nu} \mu_i c_i(x) = 0
\end{equation}
\item[\textbf{Critère de faisabilité primale}] \begin{equation}
\left\{
\begin{array}{l l}
  c_i (x) = 0    & \quad i \in \xi \\
  c_j (x) \leq 0 & \quad j \in \nu \\
\end{array} \right.
\end{equation}
\item[\textbf{Critère de faisabilité duale}] \begin{equation}
\mu_j \geq 0, j \in \nu
\end{equation}
\item[\textbf{Critère de complémentarité}] \begin{equation}
\mu_j c_j(x) = 0, j \in \nu
\end{equation}
\end{description}
\end{mydef}

Il est à noter que la Définition~\ref{def:chap1_kkt} fournit des
conditions suffisantes afin de déterminer si un point est un minimum
local. Sous conditions de régularité, et dans le cas où le problème
est convexe, les conditions KKT sont nécessaires et suffisantes.


\subsection{Zoologie des solveurs}

\subsubsection{Zoologie mathématique}


Au-delà de cette présentation de la théorie générale, on peut
regrouper les techniques de résolutions en différentes catégories
selon la difficulté des problèmes pouvant être traités. Nous ne nous
attarderons ici que sur les techniques d'optimisation
continues. Différentes caractéristiques peuvent rendre la résolution
plus délicate:
%
\begin{itemize}
\item Présence de contraintes,
\item Non-convexité de la fonction de coût ou des contraintes,
\item Non-linéarité de la fonction de coût ou des contraintes,
\item Fonction non -- ou mal -- définie en dehors des contraintes,
\item Variations numériques fortes entraînant des erreurs numériques
  difficiles à juguler,
\item etc.
\end{itemize}
%
La combinaison de ces difficultés entraînent la classification des
solveurs dans de très nombreuses catégories dont un extrait est
détaillé dans le Tableau~\ref{tbl:chap1_solver}.
%
\begin{table}\label{tbl:chap1_solver}
\begin{center}
\begin{tabular}{|>{\small}p{.24\textwidth}|>{\small}p{.305\textwidth}|>{\small}p{.305\textwidth}|}
\hline
\textit{Classe de problème} & \textit{Fonction de coût} & \textit{Contraintes}\\
\hline
\textbf{Moindres carrés linéaires} & $\sum_{i \in \{1, 2, \dotsc, n\}} (r_i - \mathbf{x}_i)^2$ où les $r_i$ sont des constantes & non \\
\hline
\textbf{LQP} \small{(opt. quadratique linéaire)} & quadratique & linéaire \\
\hline
\textbf{SQP} \small{(opt. non-linéaire)} & non-linéaire & non-linéaire \\
\hline
\end{tabular}
\end{center}
\caption{Zoologie des problèmes en optimisation numérique.}
\end{table}
%
Les algorithmes de résolution ont également des particularités
intrinsèques qui peuvent être particulièrement intéressantes dans le
cadre de la robotique. Par exemple, le démarrage à chaud ou ``warm
start'' \index{démarrage à chaud} \index{warm start|see{démarrage à
  chaud}} est la possibilité, pour un solveur, de pouvoir résoudre
plusieurs problèmes à la suite tout en se souvenant de son état
interne afin de pouvoir gagner en temps de calcul lorsque le nouveau
problème est très proche du problème précédemment résolu. Cette
situation intervient notamment lorsque l'on optimise une trajectoire
tout en l'executant. L'optimisation se déroule pour différents
instants proches dans le temps pour lesquels l'évolution de l'état du
robot est minime. Une autre caractéristique est la capacité, pour un
solveur, de pouvoir être interrompu à n'importe quelle itération tout
en guarantissant que la solution actuelle, bien qu'incomplète,
respecte la totalité des contraintes du problème. Ce comportement est
important lorsque l'optimisation est lancée dans un contexte où les
contraintes temporelles sont fortes, telle que dans le contrôleur
calculant les commandes du robot.



\subsubsection{Zoologie logicielle}


On l'aura compris: d'une théorie unique découle un ensemble
d'algorithmes pouvant résoudre des problèmes de complexité
variée. Plaçons nous désormais à la place d'un roboticien devant
résoudre un problème particulier. Quels logiciels peut-il utiliser
pour se faire? Le Tableau~\ref{tbl:chap1_soft} détaille une partie
des paquets logiciels existants.
%
\begin{table}\label{tbl:chap1_soft}
\begin{center}
\begin{tabular}{|p{.24\textwidth}|p{.305\textwidth}|p{.305\textwidth}|}
\hline
\textit{Logiciel} & \textit{Classe de problème} & \textit{Caractéristiques}\\
\hline
\textbf{\href{http://devernay.free.fr/hacks/cminpack/index.html}{C/C++ MinPack}} & Moindres carrés linéaires & C/C++, réentrant, thread-safe \\
\hline
\textbf{\href{https://projects.coin-or.org/Ipopt}{Coin IPOPT}} & SQP & C++, réentrant \\
\hline
\textbf{\href{http://www.aemdesign.com/}{CFSQP}} & SQP & C, réentrant \\
\hline
\end{tabular}
\end{center}
\caption{Zoologie des problèmes en optimisation numérique.}
\end{table}
%
Une conclusion s'impose au regard de cette liste de paquets logiciels:
les techniques les plus compliquées à implémenter sont peu disponibles
et il n'existe pas d'outil unifié permettant de résoudre différents
types de problèmes. Enfin, chaque outil traite une catégorie de
problème en particulier. Il est raisonnable de se demander si chaque
utilisateur, lorsqu'il commence à définir son problème est à même de
choisir de manière pertinente le solveur -- et donc le framework --
adapté à son problème. Est-il certain que toutes les fonctions seront
différentiables et continues deux fois? une fois? Est-on certain de ne
pas avoir besoin de contraintes inégalités, voire de ne pas avoir
besoin du tout de contraintes? Rien n'est moins sûr. Un exemple simple
est la gestion des collisions. On peut tout à fait développer un outil
robotique et le faire fonctionner dans un environnement ouvert avant
de vouloir s'attaquer à un problème plus difficile et prendre en
compte les collisions. Cela peut poser des problèmes extrêmement
important car le solveur ne sera pas le même dans les deux
cas. Cependant, il semble qu'avoir à réécrire, ou tout du moins
adapter, le problème précédent à un nouvelle bibliothèque logicielle
est inutile puisque finalement les deux problèmes se formalisent au
sein d'un même paradigme.


\section{De la dif\-ficulté à passer d'un paradigme uni\-que à une implémentation unifiée}


A première vue, il pourrait sembler que l'absence d'une architecture
unifiée pour la résolution de problèmes d'optimisation numérique est
le simple résultat d'un manque de volonté et/ou de coordination entre
développeurs. Nous allons montrer ici qu'il n'en est rien. Une simple
agglomération des implémentations de différentes stratégies de
résolution ne saurait donner un ensemble cohérent et performant en
terme de temps de calcul sans une véritable réflexion sur comment
modéliser le paradigme décrit à la section précédente tout en prenant
en compte les limites internes des langages de programmation. Nous
allons tout particulièrement nous intéresser à la programmation
orientée objet qui est la fondation d'une large proportion des
langages modernes.


\subsection{Typage, programmation orienté objet et familles de types}

\subsubsection{$\lambda$-calcul et théorie des types}

Lors du développement de l'informatique, les langages de programmation
ont tout d'abord été de simples outils permettant de représenter du
code machine de manière succinte et plus facilement lisible par les
humains. Cependant, avec la complexification croissante des
applications informatiques, vérifier la correction des programmes est
rapidement devenu un enjeu critique. Pour se faire, les langages de
programmation ont progressivement adopté le typage comme moyen de
n'autoriser que la construction de programmes valides. Cette notion de
type a issue de la théorie des types \index{théorie des types}, une
branche de la logique mathématique. La première théorie unifiée des
types a été développée par Bertrand Russel \index{Bertrand Russel} au
début du vingtième siècle dans les \emph{Principia
  Mathematica}~\citep{pm}.


Afin de démontrer les difficultés que peuvent poser la modélisation
d'un problème d'optimisation numérique, nous allons nous appuyer sur
un système de calcul formel particulier, le $\lambda$-calcul
\index{$\lambda$-calcul} inventé par Alonzo Church \index{Alonzo Church}
dans les années trente. Ce système de calcul formel a une expressivité
équivalente à une \index{Machine de Türing}. Nous allons tout d'abord
nous concentrer sur la construction des lambda-termes
\index{lambda-terme}, c'est à dire l'ensemble des expressions
syntaxiquement correcte pouvant être construite en lambda-calcul. Ces
lambda-termes peuvent être divisés en trois catégories: variables,
applications et abstractions. Les variables: $x$, $y$, \ldots Les
applications sont l'ensemble des expressions $u v$ où $u$ et $v$ sont
deux lambda-termes. Enfin les abstractions sont les expressions du
type: $\lambda x.u$ où $x$ est une variable et $u$ un
lambda-terme. Par exemple, l'application identité peut s'écrire en
lambda-calcul: $\lambda x.x$.


La procédure qui, à partir de n'importe quel lambda-terme, tente de la
simplifier en réalisant les applications par réécriture est appelé
$\beta$-conversion \index{$\beta$-conversion}. En réduisant un
lambda-terme, on réalise le calcul associé à ce dernier. Une propriété
utile serait de pouvoir déterminer si toutes les lambda-réductions
terminent, c'est à dire que l'on peut réduire tous les lambda-termes
jusqu'à un point où plus aucune réécriture \index{réécriture} n'est
possible.


Hélas, ce n'est pas le cas en lambda-calcul. Soit:

\begin{eqnarray}
  \Delta \equiv \lambda x.x\\
  \Omega \equiv \Delta \Delta \equiv (\lambda x.x) (\lambda x.x)
\end{eqnarray}

 On peut remarquer que la $\beta$-reduction de $\Omega$ boucle
 indéfiniment:

\begin{equation}
  (\lambda x.x) (\lambda x.x) =_{\beta} (\lambda x.x) (\lambda x.x)
\end{equation}

En terme calculatoire, on peut considérer que ce programme boucle et
ne termine jamais.


Face à ce constat, une autre interrogation se pose: peut-on construire
un sous-ensemble des lambda-termes normalisables? La réponse est oui,
via l'utilisation du lambda-calcul simplement typé
\index{lambda-calcul simplement typé}. Dans cette version du
lambda-calcul, toute expression est annotée par un type. Les types
sont de deux sortes: $\iota$ et $\tau_1 \rightarrow \tau_2$ à
condition que $\tau_1$ et $\tau_2$ soient des types. $\iota$
représente le(s) type(s) primitif(s) du langage permettant de
représenter les booléens ou un sous-ensemble des nombres entiers ou
réels par exemple. Si $x$ est de type $\iota$, on a alors:
%
\begin{equation}
  x \vdash \iota
\end{equation}
%
\ldots prononcé ``$\iota$ type $x$''. L'ensemble des types est quand à
lui noté $\Gamma$.


En définissant des règles de typage appropriée, on peut restreindre
les lambda-termes bien typés aux lambda-termes
normalisables. Cependant, une limitation de cette approche est qu'une
grande partie des expressions que l'on souhaiterait construire ne sont
pas correctement typés. En particulier, il n'est pas possible de
construire la fonction exponentielle dans ce paradigme.


\subsubsection{Programmation orienté objet et typage}

Indépendamment des systèmes de calculs formels, divers modèles plus
pratiques ont été développé au cours de la seconde moitié du vingtième
siècle. Notamment, la programmation orienté objet \index{programmation
  orienté objet} a été introduite par Ole-Johan Dahl \index{Ole-Johan
  Dahl} et Kristen Nygaard \index{Kristen Nygaard} dans \index{Simula
  I} au cours des années soixante. Ce langage avait pour but de
permettre le développement de simulation à événements discrets. Ce
paradigme de pensée s'est progressivement développé jusqu'à être
massivement adopté par la majorité des langages de programmation au
cours des années quatre-vingt dix. L'adoption massive du C++ (Bjarne
Stroustrup \index{Bjarne Stroustrup}, 1983) a participé à la
démocratisation de la POO.


La POO introduit la classe comme élément unitaire permettant la
conception d'un programme informatique. Alors qu'auparavant, une
séparation stricte était observée entre les données et les
algorithmes, une classe est un élément du langage regroupant à ces
deux éléments. Les algorithmes fournit par une classe sont alors
appelés ``méthodes'' tandis que les données prennent le nom
``d'attributs''. Cette réunification des données et des traitements
sur les données s'inscrit pleinement dans la lignée de la vision
informatique introduite par Von Neumann et l'architecture portant son
nom. L'innovation clé de cette architecture a été de permettre la
transmission sur un même bus des instructions composant un programme
ainsi que des données sur lesquelles les instructions vont
s'appliquer. On peut donc dire que la classe a poussé la réunification
algorithmes/données jusqu'à la conception interne des logiciels.

\begin{mydef}\label{def:chap1_type}
  Un type peut soit être un type fondamental noté $\iota$, soit un
  type fonctionnel, soit un type de classe. En pratique, il existe
  plusieurs types fondamentaux: réels, entiers, booléens mais le
  nombre de types de base ne change en rien les règles de typage.
\end{mydef}

\begin{mydef}\label{def:chap1_class}
  Soit $\tau$ un type de classe. $\tau$ est défini par $a \in \mathfrak{A}_\tau$
  l'ensemble des attributs -- données -- de la classe et $m \in
  \mathfrak{M}_\tau$ l'ensemble des méthodes de la classe.

  \emph{Notation:} on notera $\tau.a$ l'attribut $a$ de la classe $\tau$ et
  $\tau.m$ la méthode $m$ de la classe $\tau$. Afin d'éviter les ambiguïtés,
  l'ensemble des symboles composant les éléments $\mathfrak{A}_\tau$ et
  $\mathfrak{M}_\tau$ sont disjoints.
\end{mydef}

\begin{mydef}\label{def:chap1_method}
  Soit $\mathfrak{M}_\tau$ l'ensemble des méthodes d'une classe. Une
  méthode $m$ du type de classe $\tau$ est une variable de type $\tau
  \rightarrow \tau'$ où $\tau'$ est un type quelconque.

  \emph{Notation:} L'application de la méthode $m$ de la classe $\tau$
  devrait s'écrire $a.m \tau$ où $a \vdash \tau$. Dans la mesure où
  l'on sait que $m \in \mathfrak{M}_\tau$, on peut se passer
  d'appliquer explicitement le premier argument du type fonctionnel et
  directement écrire $\tau.m$ pour signifier l'application de la
  méthode $m$.
\end{mydef}

\begin{mydef}\label{def:chap1_attribute}
  Soit $\mathfrak{A}_\tau$ l'ensemble des attributs d'une classe. Un
  attribut $a$ du type de classe $\tau$ est une variable de type
  $\tau'$ quelconque à l'exception des types fonctionnels prenant un
  type $\tau$ en argument.
\end{mydef}


\begin{myexample}\label{ex:chap1_class}
  Soit $A$ une classe définie par:

  \begin{equation}
    A \equiv \{ \underbrace{\{ \text{foo} \vdash \iota
      \}}_{\text{attribut}}, \underbrace{\{ \text{bar} \vdash A
      \Rightarrow \iota, \text{baz} \vdash A \rightarrow \iota
      \}}_{\text{méthodes}} \}
  \end{equation}

  $A$ possède un attribut et deux arguments. D'après la définition de
  $A$ et les
  Définitions~\ref{def:chap1_method}~et~\ref{def:chap1_attribute}, on
  a:

  \begin{equation}
    a \vdash A \Rightarrow a.\text{foo} \vdash \iota
  \end{equation}

  \begin{equation}
    a \vdash A \Rightarrow a.\text{bar} \vdash A \rightarrow \iota
  \end{equation}
\end{myexample}


Cette représentation permet de pouvoir offrir une vue d'ensemble
intégrant à la fois des données structurées et l'ensemble des
opérations possibles sur ces dernières. Auparavant, la séparation de
ces deux éléments rendait plus difficile la construction d'une vision
globale: on ne pouvait, par définition, pas connaître la liste des
opérations définies sur un groupe de données. Soit $\tau$ un type de
classe, l'ensemble des fonctions prenant $\tau$ en entrée n'est pas
connu lors de la compilation car il peut être augmenté à tout moment
d'un nouvel élément en définissant une nouvelle routine. Il convient
donc de documenter la liste des opérations disponibles.

Inversement, il est souvent nécessaire d'augmenter le comportement
d'une classe en y ajoutant de nouvelles données et/ou de nouveaux
algorithmes. La POO modélise ce processus au travers de l'héritage de
classe. Soit $\tau_1$ un type de classe, $\tau_2$ un type de classe
héritant de $\tau_1$. On dit également que $\tau_2$ est un sous-type
de $\tau_1$, et est noté sous la forme:

\begin{equation}
  \tau_2 <: \tau_1
\end{equation}
%
%
Cette notion, proche de celle d'inclusion en théorie des ensembles,
permet de définir des familles de types interchangeables.
%
%
\begin{mydef}\label{chap1_oo_app}
  Soit $\Gamma$ un contexte de type. $\tau_1$, $\tau_2$, $\tau_3$
  trois types appartenant à ce contexte. La règle de typage des
  applications en programmation orienté objet est alors:


  \begin{equation}
    u~v~\text{bien formé} \Rightarrow u \vdash \tau_1 \rightarrow
    \tau_2 \wedge \tau_3 <: \tau_1
  \end{equation}

  Dès lors que $tau_3$ est un sous-type de $\tau_1$, les fonctions
  prenant en argument une variable de type $\tau_1$ acceptent
  également les arguments de type $\tau_3$.
\end{mydef}
%
\begin{mydef}\label{chap1_method_typing}
  Soit $\Gamma$ un contexte de type. $\tau_1$, $\tau_2$ deux types.

  On dit que $\tau_2 <: \tau_1$ si et seulement si:

  \begin{equation}
    \tau_2 <: \tau_1 \Leftrightarrow \mathfrak{A_{\tau_1}} \subset
    \mathfrak{A_{\tau_2}} \wedge \mathfrak{M_{\tau_1}} \subset
    \mathfrak{M_{\tau_2}}
  \end{equation}
\end{mydef}
%
Un sous-type se doit de contenir au moins tous les attributs et
méthodes de la classe dont elle hérite afin d'interdire la réécriture
vers des termes syntaxiquement faux. Le ``principe de substitution de
Liskov''\index{principe de substitution de
  Liskov}~\citep{Liskov94familyvalues} définit, de manière équivalente,
la relation de sous-typage sous l'énonce suivant:
%
\begin{mydef}\label{chap1_liskov}
  Si $q(x)$ est une propriété démontrable pour tout objet $x$ de type
  $T$, alors $q(y)$ est vraie pour tout objet $y$ de type $S$ tel que
  $S$ est un sous-type de $T$.
\end{mydef}


La Définition~\ref{chap1_method_typing} introduit implicitement la
notion de polymorphisme d'héritage\index{polymorphisme!polymorphisme
  d'héritage}. Le polymorphisme\index{polymorphisme} est la capacité
pour un type fonctionnel de ne pas accepter en entrée un type unique,
mais une famille de types. Le polymorphisme le plus simple pouvant
être défini est la surcharge\index{surcharge}. Ce méchanisme autorise
la création de fonctions se réécrivant vers des $\lambda$-termes
différents selon le type de l'argument d'entrée. La formalisation de
ce système est décrit dans~\citep{Castagna95acalculus}.

Une autre forme de polymorphisme est le polymorphisme d'héritage. Si
$\tau_1$ et $\tau_2$ sont deux types tels que $\tau_2 <: \tau_1$ et
qu'il existe une méthode $m$ dans $\tau_1$ et $\tau_2$, alors tout
application de la méthode $m$ à un objet de type $\tau_2$ se réécrira
vers le $\lambda$-terme $\tau_2.m$. Un sous-type peut donc fournir une
réécriture alternative d'une fonction afin d'en spécialiser le
comportement.

La généralisation de cette propriété à toutes les fonctions au lieu
des méthodes uniquement permets d'obtenir le ``multiple
dispatch''~\citep{Muschevici2008}. Ce méchanisme est disponible dans
un nombre assez limité de langages de programmation dont Common
Lisp. En effet, contrairement à la surcharge, le multiple dispatch
ainsi que le polymorphisme d'héritage nécessite une résolution à
l'execution qui peut pénaliser l'implémentation des algorithmes. En
limitant la résolution dynamique au type du premier argument des
méthodes, on limite également le coût de la résolution à
l'execution. Ceci explique en partie l'une des raisons pour lesquelles
le premier argument des méthodes est omis dans de nombreux langages de
programmation. On ``masque'' par ce biais, le caractère spécifique de
la résolution sur le premier argument de la méthode et on peut, une
fois ce premier argument omis, considérer qu'il n'y a pas de
résolution dynamique sur les arguments des méthodes.


\subsubsection{Types paramétrés et génération de types}

Face au coût induit par la résolution dynamique du polymorphisme par
héritage, on peut se demander s'il existe un moyen d'arriver au même
résultat sans ralentir l'execution d'un algorithme. Une solution est
de générer des types afin de pouvoir déterminer à la compilation la
résolution des fonctions polymorphes. Ce méchanisme est appelé
``templates'' en C++ et se formalisent en $\lambda$-calcul sous la
forme de types paramétrés.


\begin{mydef}\label{chap1_parametrized_type}
  Un type paramétré est une fonction d'ordre supérieure $\text{type}
  \Rightarrow \text{type}$.

  \emph{Notation:} Soit $\tau$ un type paramétré. Le type $\tau$
  paramétré par $\alpha$ est dénoté $\tau_\alpha$. Il représente
  l'application du type $\alpha$ à la fonction d'ordre supérieure
  $\tau$.
\end{mydef}

Les types paramétrés permettent de définir un nouveau type de
polymorphisme, le polymorphisme
paramétré.\index{polymorphisme!polymorphisme paramétré} Ce
polymorphisme peut se formaliser en rendant les méthodes des classes
paramétrées dépendante du paramètre de classe.

En pratique, ce type de polymorphisme est particulièrement intéressant
car il n'induit pas de coût à l'execution.


\subsection{Modélisation informatique du problème}

La section précédente a introduit le formalismes et les outils que
nous allons utiliser pour poser le problème de manière formelle.


Un solveur est un algorithme qui prend en entrée un problème
d'optimisation et qui calcule un point dans l'espace des solutions.
Nous allons noter $\iota$ le type primitif représentant une partie des
réels et $[\iota]$ une liste de zéro, un ou plusieurs variables de
type $\iota$. $\tau_{\text{prob}}$ le type représentant le problème
d'optimisation. Nous allons voir dans la section suivante comment ce
type est habituellement défini, introduire une paramétrisation
alternative du problème fournie par RobOptim et enfin montrer quelles
avancées cette nouvelle modélisation engendre.  Nous allons travailler
en utilisant un solveur générique qui sera dénoté:


\begin{equation}\label{eq:chap1_prob_type}
  \text{solv} \vdash \tau_{\text{prob}} \rightarrow \iota
\end{equation}


La vision habituellement implémentée consiste à définir un unique type
$\tau_{\text{prob}}$ dédié à un solveur précis. L'approche proposée
ici consiste, à la place, à définir une famille de types formées par
des classes, sous-classes éventuellement paramétrées pour tenter
d'exprimer la totalité de la théorie de l'optimisation numérique sous
la forme d'une et une seule modélisation.


\subsubsection{De la fonction\ldots au type fonctionnel?}


Naïvement, on pourrait penser qu'il n'y a rien de plus simple à
modéliser qu'une fonction mathématique. En effet, les types
fonctionnels ne sont-ils pas une transcription directe des fonctions
mathématiques habituelles? On aurait donc, à première vue:

\begin{equation}
  \begin{array}{ccccc}
    f & : & \tau_1 & \rightarrow & \tau_2\\
    ~ & ~ & x      & \mapsto     & f(x)
  \end{array}
  \xhookrightarrow{\text{devient}}
  f \vdash \tau_1 \rightarrow \tau_2
\end{equation}

Cette modélisation convient, bien sûr, mais ne permet pas de ranger
les catégories en fonction selon leur nature: continue,
différentiable, etc.

En effet, pouvoir accéder au gradient de $f$ est primordial, mais
implémenter une fonction $\mathbf{\nabla}(f)$ est difficile car dans
de nombreux langages, dont celui utilisé par RobOptim -- C++ --, les
fonctions ne sont pas des objets de premier ordre et leur manipulation
est, de fait, mal-aisée.

Afin de pouvoir associer des méta-données aux fonctions, une
représentation naturelle est l'utilisation d'une hiérarchie de classe.

Le critère choisi est la classe de la fonction: $\mathcal{C}^0$,
$\mathcal{C}^1$, \ldots à une nuance près. En effet, si une fonction
de classe $\mathcal{C}^n$ est déclarée, elle doit non seulement être
dérivable $n$ fois et continue $n+1$ fois, mais les valeurs des $n$
fonctions dérivées doivent pouvoir être calculées.

La classe de fonction associée à la possibilité de pouvoir calculer
les gradients est un facteur très important lors de l'évaluation du
``niveau'' de difficulté d'un problème d'optimisation. Dès lors,
regrouper les fonctions selon ce critère semble être un choix
raisonnable.

\begin{mydef}\label{def:chap1_fcn}
  Soit $f$ une fonction mathématique continue dont on ne sait pas
  calculer la dérivée. Cette catégorie de fonctions est définie par la
  classe $\text{Fonction}$ définit par:

  \begin{equation}
    \text{Fonction} \equiv \{ \underbrace{\{ \#\text{entrée} \vdash
      \iota \}}_{\text{attributs}}, \underbrace{\{ \text{calcule}
      \vdash [\iota] \rightarrow \iota \}}_{\text{méthode}} \}
  \end{equation}

  L'attribut $\#\text{entrée}$ la cardinalité de l'espace d'entrée de
  la fonction et la méthode $\text{calcule}$ permets d'évaluer la
  fonction en un point. Nous ne traiterons ici que les fonctions dont
  l'espace de sortie est de cardinalité $1$. En effet, les fonctions
  dont l'espace de sortie sont de cardinalité $n$ peuvent être
  facilement représentables par $n$ fonctions dont l'espace de sortie
  est de taille $1$.
\end{mydef}

\begin{mydef}\label{def:chap1_derivfcn}
  Soit $f$ une fonction mathématique continue et dont on sait calculer
  la dérivée une fois. Cette catégorie de fonctions est définie par la
  classe $\text{FonctionDérivable}$ définit par:

  \begin{equation}
    \text{FonctionDérivable} \equiv \{ \{ \}, \{ \text{gradient}
    \vdash [\iota] \rightarrow [\iota] \} \} <: \text{Fonction}
  \end{equation}

  Le type $\text{FonctionDérivable}$ hérite du type $\text{Fonction}$
  et fournit, de surcroît, la méthode $\text{gradient}$ permettant
  d'évaluer le gradient de la fonction en un point.
\end{mydef}

\begin{mydef}\label{def:chap1_nderivfcn}
  Soit $f$ une fonction mathématique continue et dont on sait calculer
  la dérivée $n$ fois. Cette catégorie de fonctions est définie par la
  famille de classe $\text{Fonction}_n$ définit par:

  \begin{equation}
    \text{Fonction}_n \equiv
    \left\{
    \begin{array}{l l}
      \text{Fonction} & \quad \text{si $n = 0$}\\
      \text{FonctionDérivable} & \quad \text{si $n = 1$}\\
      \{ \{ \}, \{ \text{gradient} \vdash \iota \rightarrow [\iota] \rightarrow [\iota] \} \} <: \text{Fonction}_{n-1} & \quad \text{si $n > 1$}\\
    \end{array} \right.
  \end{equation}

  On définit de sorte une famille de types formant une hiérarchie de
  classe sous la forme d'une peigne où $\mathcal{C}^n <:
  \mathcal{C}^{n-1} <: \dotsc <: \mathcal{C}^1 <: \mathcal{C}^0$.

  A partir de $\mathcal{C}^2$, on remarquera que la méthode
  $\text{gradient}$ est surchargée selon le nombre d'argument. Avec un
  argument de type $[\iota]$, le gradient est évalue en un point. Avec
  deux arguments $o \vdash \iota$ et $\mathbf{x} \vdash [\iota]$, on
  aura pour $o \leq n$ le résultat de l'évaluation de la dérivée d'ordre
  $o$.
\end{mydef}


Cette hiérarchie de classe est le c\oe ur de la modélisation
informatique des problèmes d'optimisation car elle permets d'exprimer
à la fois la fonction de coût et les contraintes éventuellement
présentes dans un problème. Il suffit alors à l'utilisateur souhaitant
définir sa fonction $F$ de le faire hériter du type adéquat de la
famille $\text{Fonction}_n$.

Cette approche sous forme de hiérarchie linéaire de classes en forme
de peignes à toutefois des inconvénients qui seront détaillés dans la
dernière section de ce chapitre.


\subsubsection{Des problèmes divers, un type unique}


Nous avons donc toute une famille de types définissant des fonctions
mathématiques: non seulement les $\text{Fonction}_n$ qui modélisent
les interfaces de notre programme, mais aussi tous les types
correspondant aux fonctions présentes dans un problème d'optimisation
particulier.

La question est désormais comment modéliser une ``classe'' de problème
d'optimisation numérique de manière générique tout en assurant une
sûreté maximum par typage.

La proposition réalisée ici est de définir une famille de types
$\tau_{\text{prob}}$ paramétré par $n > 0$ type(s). Le premier définit
le type de la fonction de coût tandis que tous les autres définissent
les types de fonctions des contraintes.


\begin{mydef}\label{def:chap1_prob}
  Soit $\tau_{\text{prob}(\tau_1, \tau_2, \tau_3, \dotsc)}$ tel que
  $\tau_i <: \text{Fonction}$, $i \in \{1, \dotsc, n\}$ par:

  \begin{equation}
  \begin{split}
    \tau_{\text{prob}(\tau_1, \tau_2, \tau_3, \dotsc)} \equiv
    \{ \{ & \text{coût} \vdash \tau_1,\\
    & \text{contraintes} \vdash [\cup_{i \in \{2, \dotsc, n\}} \tau_i],\\
    & \text{bornes\_contraintes\_min} \vdash [\iota], \\
    & \text{bornes\_contraintes\_max} \vdash [\iota] \},\\
    & \{ \} \}\\
  \end{split}
  \end{equation}

  Un problème d'optimisation est constitué d'une fonction de coût de
  type $\tau_1$ et un ensemble de fonctions pouvant être de type
  $\tau_i$, $i \in \{2, \dotsc, n\}$. Cet ensemble de fonctions
  associé à un ensemble d'intervalles permettent de définir des
  contraintes égalité ou inégalité. Soit $\text{contraintes}_i$ la
  $i$-ème contrainte du problème, on a:

  \begin{equation}
    \text{bornes\_contraintes\_min}_i \leq \text{contraintes}_i \leq
    \text{bornes\_contraintes\_max}_i
  \end{equation}

  On notera que construire une contrainte égalité revient à spécifier
  une borne minimum et maximum identique.
\end{mydef}


La Définition~\ref{def:chap1_prob} permet donc de modéliser un
problème d'optimisation numérique via l'utilisation d'un type
paramétré générique. Cette modélisation couvre une large partie des
problèmes d'optimisation avec ou sans contrainte tout en fournissant
une interface unique. Nous sommes donc dans un cas où l'utilisateur
peut définir son problème de manière naturelle, sans avoir à partir
d'un solveur qui impose son formalisme spécifique.


\subsubsection{Le chaînon manquant: le solveur}


Evidemment, il ne saurait être question de terminer cette section sans
revenir une dernière fois sur le c\oe ur et l'âme du système: le
solveur. Dans la section précédente, nous avons construit un type
paramétré qui satisfait les contraintes énoncées au début de cette
section, voire l'\'Equation~\ref{eq:chap1_prob_type}.


En effet, on peut construire un type paramétré par le type de problème
permettant de modéliser un solveur.

\begin{mydef}\label{def:chap1_solver}
  Soit $\text{Solveur}_{\tau}$ une classe paramétrée par un type $\tau$
  tel que $\exists n, \exists \tau_1, \dotsc, \tau_n, \tau <:
  \text{Problème}_{\tau_1, \dotsc, \tau_n}$ et défini par:

  \begin{equation}
    \text{Solveur}_{\tau} \equiv \{ \{ \}, \{ \text{résoud} \vdash
    \tau \rightarrow [\iota] \} \}
  \end{equation}
\end{mydef}

La Définition~\ref{def:chap1_solver} fournit la dernière famille de
types indispensables à la modélisation d'un problème d'optimisation
numérique. Nous allons désormais nous attacher à montrer que diverses
classes de problèmes d'optimisation peuvent être exprimées via cette
modélisation unique.


\section{Exemples de formalisation de classes de problème}


Afin d'illustrer de démontrer que le formalisme décrit dans la section
précédente peut réellement permettre de décrire des problèmes
d'optimisation de manière efficace et générique, nous allons
construire les types représentant trois classes de problèmes en
utilisant les familles de types décrites dans la section
précédente. Les trois classes choisies sont:

\begin{itemize}
\item moindres carrés linéaires
\item optimisation non-linéaire: fonction de coût non-linéaire,
  contraintes linéaires ou non-linéaires. Le gradient des fonctions
  doit pouvoir être évalué.
\item optimisation non-linéaire: fonction de coût non-linéaire,
  contraintes linéaires ou non-linéaires. Le hessien\index{hessien}
  des fonctions doit pouvoir être évalué (dérivée d'ordre 2).
\end{itemize}


\subsection{Moindres carrés linéaires}

Les moindres carrés linéaires utilisent une fonction de coût suivant
une structure très particulière. Proposer un type modélisant cette structure est possible:

\begin{mydef}\label{ex:chap1_linearfct}
  Soit $\text{FonctionLinéaire}$ un type tel que
  \mbox{$\text{FonctionLinéaire} <: \text{Fonction}_2$}.
\end{mydef}

Nous avons ici choisi de faire dériver les fonctions linéaires du type
$\text{Fonction}_2$. Cela ne correspond pas à la réalité mathématique
dans la mesure où une fonction linéaire est infiniment dérivable, la
dérivée étant nulle à partir de la second dérivation. Il n'est pas
possible d'exprimer cet état de fait au regard de la modélisation
adoptée ce qui représente une limite du modèle de typage proposé. En
pratique, les fonctions plus de deux fois dérivables ne présentent pas
d'intérêt particulier et cette limite ne pose donc pas de problèmes en
pratique.

\begin{mydef}\label{ex:chap1_leastsquarefct}
  Soit $\text{MoindreCarré}$ un type tel que
  \mbox{$\text{MoindreCarré} <: \text{FonctionLinéaire}$}.

  Pour rappel, la fonction s'exprime mathématiquement sous la forme:

  \begin{equation}
    \sum_{i \in \{1, 2, \dotsc, n\}} (r_i - \mathbf{x}_i)^2
  \end{equation}
  \ldots où les $r_i$ sont des constantes.

  Cette fonction peut se modéliser par le type suivant:
  \begin{equation}
    \text{MoindreCarré} \equiv \{ \{ r \vdash [\iota] \}, \{ \} \} <: \text{FonctionLinéaire}
  \end{equation}
  où l'attribut $r$ représente les $r_i$ de l'équation précédente.
\end{mydef}


Une fois ces types définis, il ne reste plus qu'à instantier les types
paramétrés avec un ensemble de paramètres adéquats.

\begin{myexample}\label{ex:chap1_leastsquare}
  Paramétrisation proposée:
  \begin{description}
  \item[Fonction de coût] $\text{MoindreCarré}$
  \item[Contraintes] non
  \item[Problème] $\text{Problème}_{\text{prob}(\text{MoindreCarré})}$
  \end{description}
\end{myexample}


\subsection{Optimisation non-linéaire (gradient uniquement)}

Nous allons traiter ici la classe de problèmes suivante: la fonction
de coût est non-linéaire et les contraintes séparées en deux types
distincts: linéaires ou non-linéaires.

\begin{myexample}\label{ex:chap1_cfsqp}
  Afin de pouvoir décrire cette classe de problème, la paramétrisation des types choisie est la suivante:
  \begin{description}
  \item[Fonction de coût] $\text{Fonction}_1$
  \item[Contraintes linéaires] $\text{FonctionLinéaire}$
  \item[Contraintes non-linéaires] $\text{Fonction}_1$
  \item[Problème] $\text{Problème}_{\text{prob}(\text{Fonction}_1, \text{FonctionLinéaire}, \text{Fonction}_1)}$
  \end{description}
\end{myexample}

\subsection{Optimisation non-linéaire (gradient et hessien)}

Si l'on définit la même classe de problème mais en ajoutant la
possibilité de pouvoir évaluer les hessiens des fonctions
non-linéaires, on obtient la paramétrisation suivante.

\begin{myexample}\label{ex:chap1_ipopt}
  Paramétrisation proposée:
  \begin{description}
  \item[Fonction de coût] $\text{Fonction}_2$
  \item[Contraintes linéaires] $\text{FonctionLinéaire}$
  \item[Contraintes non-linéaires] $\text{Fonction}_2$
  \item[Problème] $\text{Problème}_{\text{prob}(\text{Fonction}_2, \text{FonctionLinéaire}, \text{Fonction}_2)}$
  \end{description}
\end{myexample}


\section{Discussion}


Dans les sections précédentes, un formalisme permettant de représenter
de manière unifiée les problèmes d'optimisation numérique a été
présenté. On peut se demander quel apport l'utilisation d'un paradigme
unique peut apporter. Dans la plupart des cas, les utilisateurs de
solveurs numériques partent d'un outil existant et tentent de
modéliser le problème à résoudre de sorte qu'il puisse être soluble
par l'outil choisi. Cette approche est criticable car chaque problème
possède une difficulté intrinsèque et l'on devrait au contraire
pouvoir dans un premier temps modéliser son problème puis choisir le
solveur adéquat une fois la modélisation correctement effectuée. De
plus, le travail de recherche comporte, par nature, une grande part de
prototypage rapide où pouvoir tester de nouvelles approches rapidement
est important. Dans ce type de situation, utiliser le solveur le plus
avancé de l'\'Etat de l'Art n'est pas toujours nécessaire et l'on peut
éventuellement choisir à dessin un solveur ``trop'' puissant pour le
problème à résoudre et sous-optimal en terme de temps de
calcul. Inversement, il est courant qu'après avoir commencé un travail
de recherche, on se rende compte qu'un solveur équivalent plus rapide
ou présentant de meilleures caractéristiques existe. Dans ce cas, il
est plus simple de l'intégrer à un formalisme existant plutôt que de
devoir adapter son problème à un logiciel alors qu'algorithmiquement
parlant il n'existe pas de différence notable. Pour terminer, il est
arrivé que l'on commence tout d'abord par poser un problème robotique
pour s'apercevoir qu'il serait bien mieux traité à l'aider d'un
solveur dédié. Il est courant donc de devoir écrire un nouveau
solveur, ou modifier un solveur existant, afin de pouvoir rendre le
traitement du problème plus rapide. Dans ce cas également, un
formalisme unique présente de nombreux avantages, notamment en terme
de temps de développement.





limites: differentiation d'une var, explosion combinatoire du nombre
de types si plusieurs criteres méthodes virtuelles, union de types,
interfaces

passage d'une classe de problème à un autre pas directe, il faut reconstruire type problème avec les bons types



\section{RobOptim: une suite logicielle pour l'optimisation numérique en robotique}



FIXME FIXME FIXME




\section{Architecture logicielle de Roboptim}

test

\section{Scénarii d'utilisation}

test

\subsection[FIXME]{\'Ecriture d'une application robotique complexe: optimiser la trajectoire de marche pour un robot humanoïde}

test

\subsection{Intégration d'un solveur tiers et extensibilité de RobOptim}

test

\section{Discussions}

test

\section{Conclusion}

test
