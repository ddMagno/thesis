\chapter{Introduction}
\label{chap:intro}

\epigraph{Text text text text text text text text text text text text
  text text text text text text text text text text text text text
  text text text text text text text text text text text text text
  text text text text text text text text text text text text text
  text text text text}{Some Author}
\clearpage

\section{Les enjeux de la robotique}

\lettrine[lines=2, lraise=0.1, nindent=0em, slope=-.5em]%
{P}{armis} les secteurs liant recherche scientifique et innovation
technologique, la robotique fait figure d'enfants prodigue. Alors ques
les études se multiplient pour annoncer le ``boom'' du secteur FIXME,
les applications possibles pour les robots semblent chaque jour plus
nombreuses: drones militaires ou aviation civile, voitures
automatisées, sondes spatiales, sous-marins autonomes et dans un futur
plus lointain l'aide aux personnes âgées.

\subsection{50 ans de robotique mobile}
\subsection{De la voiture autonome à l'assistance aux personnes âgées}

\section{Robotique humanoïde: l'enfance...}

\subsection{...d'un domaine scientifique...}
\subsection{...et d'un secteur industriel}

\subsection{Le projet japonais HRP}

\section{De la génération de mouvements à l'exécution}

